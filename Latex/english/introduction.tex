\chapter{Introduction}

\paragraph{How this idea came about}
About July, 2010 I was reading an article titled
``How to write mathematics'' by Paul Halmos\cite{halmos_teach_math}.
More or less at the same time I was watching the series
Into the Universe narrated by Stephen Hawking, plus 
I was very dissapointed with my progress in my personal 
study of mathematics.\par

\indent
It all begun when while I was reading Halmos article
I realized that many of the ideas the article mentioned could
be implemented through a computer program in order 
to save time and effort to math students.\par

\paragraph{Halmos's principles for writting maths}
\indent
Some of the basic ideas from that article are:

\begin{itemize}
 \item Writting self-contained materials
 \item The Spiral plan: iteratively improving the material
 \item Triptic: Say what you're going to say, say it, 
    say what you said.
\item Defined algorithm to determine the order in which
the topics should be presented
\end{itemize}

\indent
The key was to make the process of organizing the topics of 
a given course in a way that they are unobstrusive to the 
student. Thus allowing him to concentrate in the topic at hand.
\par

\paragraph{ Extending the principles to all sciences}

\indent
I have been a fan of science popularizers for about two or three 
years now, and like many of them, I know that ``our future depends
powerfully on how well we understand this Cosmos''.
\par

\subparagraph{Too moral- Change}
\indent
In these times, when people is obsessed with wanting to prescribe
what other people do with information, it is desperating sometimes
that the access to public knowledge be so damn difficult. Non-free Books, 
journals, courses; patents, copyright, and the like limit our capacity
to benefit from the work of others. And then you have to add the 
inherent perceived disorganization of knowledge.
\par

\indent
Although knowledge is very well organized in clumps of 
related concepts, that is generally not the case when it 
comes to non-trivial relations among concepts in what apparently
are separated fields of knowledge.
\par

\subparagraph{ Extending ... continued}
\indent
In order to preserve all the knowledge, we have to actively
preach the importance of it. So by making available to everyone
we should be in better shape.
\par

\indent
Halmos' ideas can be applied analogoustly to other sciences but
not without effort since formal sciences work in quite a different
manner from the natural ones. Mainly in their method of findind 
truths is different, natural sciences are inductive while formal 
sciences are deductive.

\indent
Nevertheless it's worth exploring 
the possibilities of following these general guidelines to attain
a better organization of our current knowledge.
\par

\indent
So, I started writting down a simple design of how could be
achieve this taking advantage of the technology of the day, 
that is, how to have all the knowledge of all the sciences
well-organized, with what for any practical purpose is a centralized
point of access, the Internet.
\par

\paragraph{The basic idea}
\indent
The basic idea is to have a system that allows you to 
find a topic or concept which you want to learn about, you 
provide a profile based on you current knowledge of the
topic and the system is then able to compute the minimal
tree of dependencies, that is, the mininal number of concepts
that are needed for you to learn before you are able to 
actually understand the selected topic or concept.
\par

\paragraph{The ideal case: Maths}

Why are the formal sciences the ideal case? Well it's is because
in mathematics is basically structured in the following manner:
you have a bunch of definitions, a bunch of axioms, and then
you use those two with well defined rules of inference to build
theorems.

\indent
A theorem is a proposition of the form, ''if A, then P´´ where A is one
or many premises and P is a conclusion that follows from A, although
this may not always be obvious.

\indent
There are special kinds of propositions
which have also the form of a theorem, what separates the theorems from 
the other propositions is that a proof for it has been given.
If no proof is given then they are called conjectures, propositions, 
or hypothesis.

\indent
In math, given a proposition the objective is to find a proof for it, to do that
is is possible to use the premises of that proposition and all the 
other definitions, axioms and theorems that have been proved so far.
Of course you won't always need all the premises, axioms and definitions to 
prove a given theorem. In fact in mathematics it's \textbf{always?} 
possible to find a minimal set of premises, axioms and theorems 
to prove a given theorem. 
And that's what makes math the ideal case for a prototype of a network 
of knowledge.

\indent
Given a concept, you can easily trace back the dependencies for it.
In contrast, it is more difficult to organize knowledge in this way
for natural sciences. Definitions change, we have to distinguish 
between causation (A implies B) and correlation (A is linked to B, but 
in a way that we don't understand yet), facts have to be interpreted
in the context of what's known,.... well you get the idea. The 
structure for knowledge in the formal sciences is more static/constant
than the structure for natural sciences, thus easier to organize.


\subsection{About generalizations}
\indent
Throughout the history of maths, and in general of science, 
there have been various instances in which a given set of 
concepts that seemed to be unrelated could later be unified in 
a generalization. This is in part the objective of this
computer system, in a not so ambitious idea, to help us 
identify which concepts seem to have some pattern that suggest
they may be related. This in principle could be done by analyzing
the structure of the network they form.
A more ambitous idea would be to create algorithms to propose
unifications/generalizations of structures that follow a
similar patter in the way they are interconnected.
 
\indent
The hypothesis is that the structure of the interconnectedness between
concepts can tell us something about an underlying principle that could
in principle unify them.


\subsection{About automation}
If it is possible to reliably find patterns in the structure
of knowledge that is in the network, then it may also be
possible to automate the proposal of new theories, automatic
proof of proposed theorems, etc.

\subsubsection{About applications}
\indent
Concepts are also connected with their \textit{applications},
that is, if an abstract concept can be used practically in 
a given scenario then that use is an application of the 
concept. For example, prime numbers have their applications
in the field of cryptography, the theory of evolution has 
repercusions in the design of new drugs, etc.

\indent
We can then also extend the network of concetps to include 
their applications. This way, we can hypothesize that 
analyzing the structure of such relations  concept-application
we may desing algorithms that coudl propose applications for 
given concepts. In principle automatically speeding up the 
proposal of applications.

\subsubsection{Why we need to speed up applications proposals}
\indent
Some believe we're are at an unprecented point of human history
in the sense that science technology had never advanced so fast as it 
is advancing now (which is true in a sense).
I personally believe that this is not enough. I thinks we 
are rather slow in developing new theories and technologies 
that could allow us to get a better understanding of the cosmos.
We have a lot of ideas that we cannot currently implement becasue
nobody has come up with the technology to do it. Some examples 
of this are, interstellar travel, efficient and clean energy 
production, anti-aging technologies, etc.


\indent
By integrating all the current knowledge in a unified computer 
system freely accessible to anyone with internet access, and 
building a social network around science research, advancing 
collaboration instead of competition, publishing publicly new
knowledge, we have a better chance of coming up with a way to 
accelarate the development of our civilization.


\subsubsection{The future of science}
\indent
In the light? of these hypothesis we can envision that 
mathematicians, and scientists in general should find ways 
to delegate research, at least in the traditional sense to 
computer systems, given that it is possible to automate
the research, hypothesys generation, and applications proposal 
process.
 
 \indent
Our efforts should be focused on finding better, and faster 
ways to advance knowlege generation, organization and 
unification.




\subsection{Ojectives}
\begin{itemize}
 \item Minimize the time and effort for learning a 
given concept.
\item Minimize the space needed to store 
information. (Reduce multiplicity)
\item Optimize the organization of knowledge.
\end{itemize}