\chapter{Definitions and conventions}

\section{From information to Knowledge}
\paragraph{Information}
\indent
We will regard as information anything that can be
represented in some manner.
\par

\indent
The previous definition is very simple, and ambiguous.
But in this case, ambiguous is good since it allows
us to extend the concept to practically everything.
\par

\paragraph{Concept}
\indent
We will use the dictionary defition of the word concept, 
which is: ''a single meaning of a term´´.
\par

\paragraph{Knowledge}
\indent
As a philosophical concept, the definition of knowledge is, and
will probably be under a lot of debate. Nevertheless we can
propose a practical definition for the sake of our pursposes.
Let's call knowledge a composite set, a set of concepts, 
as defined here, and a set of relationships between those concepts.
\par

\indent
We can represent knowledge as a network in which the nodes represent
the concepts related to a given area of knowledge, and the 
links between the nodes as the relationships between interconnected
concepts.
\par

