\subsection{The problem of information}

\noindent
The problem of information can be defined as follows: 
How to store/organize useful information 
(knowledge in the form of information and relations) 
in the best possible way?
\par

\indent
Today when we have a lot of digital information
and we want to organize or at least free some space
in our storage devices, we simply back up everything.
That is exactly the opossite thing to do in order
to solve the problem of information and I will try 
to explain why that is so.
\par


\paragraph*{Exponential growth vs. finite storage}
\indent
We know that the growth of information
can be described better by means of an exponential 
function rather than a linear function. We know
that our ability to fit more data into the same space 
(improve storage technology) has a theoretical limit 
and a technological one.
\par

\indent
So, we have in the one hand, something that grows faster
as time passes, against something that is constant in the limit
( i. e., the theoretical storage limit is a constant).
\begin{Large}\textbf{Is this true?}\end{Large}
\par


\section{Problem: Organization of knowledge}

\indent
The span of recorded history is of about five thousand years
 \cite{wiki:ancient-hist} and we have generated an inmemse 
 output of information.
 
 \paragraph{Multiplicity}
 Different cultures have arrived to the same concepts thus
 creating multiplicity.
 
 
 Multiplicity being, at least in part, a consequence of the 
 inability to publicly share information easily, this in 
 turn due to lack of technology.
 
 \subparagraph{Distributed nature of knowledge}
 It is a common scenario that a student trying to undertand
 a given topic has to resort to dive into the possbible 
different references has have been advised to check. 

Often, he ends up taking different parts from different
references, be it because of the quality of exposition, 
coverage of the topics in the different references, etc.

So, the distributed nature of the way we organize knowledge
today is a burden when we need to connect certain concepts
or when we want to keep different parts of the approach to 
the topic by the different references.

In any case, this is a waste of time, since as long as the
discussed topic is a mature topic (meaning that the theory, 
facts, etc. has been established for a fair amount of time), 
there should be a standar syllabus for addressing it given 
a predefined objetive. Topics can be studied in different 
ways depending on the final objetive, or reason of why the
concept is studied.


 
 \paragraph{The Internet's role in the organization of knowledge}
 Organizing  the information of a country/state/region 
is not an easy task, let alone organizing all the information
in the world. Nevertheless, the advances in comunication 
technologies makes it possible to share information almost
effortlessly through the internet.

\paragraph{Reducing the problem: Organizing knowledge}
This is the time when we can take advantage of the already 
created infrastructure to build upon it a system that 
enable us to organize all the information in the world.

Of course the easiest way to procced in order to do that is
to modularize the problems, so let's start not with the whole
knowledge of the world, let's organize that which we will call
in this document 'knowledge'.

Knowledge is in itself different from  infomation in the sense
that information is any data that we can think of, while knowledge
is pieces of data connected through relashionships. We may have to 
establish our own 'practical' definitions so to not deviate
in philosophical questions on what is data, information, knowledge, 
wisdom and so on.

So, let's agree that knowledge consists of a set of concepts 
which hold a relation among each other, this relation can be
of one of many possible types.

\paragraph{Ideal case: Mathematics}
To simplify even more the approach we will begin with the 
the formal sciences, Mathematics. The reason as to why start
with it can be answered rather easily, but I will elaborate 
on that to make things real clear.

Mathematics concerns itself with definitions, axioms and their 
consequences. You take a bunch of definitions, propose, verb??
some axioms and then use a set of strictly defined rules (operational 
semantics) to derive implications and other types of relations.

Once you derive a consequence, you can call it a theorem, which 
once demonstrated becomes an axiom.

Now, in mathematics it is comparatively easy to decide which axioms, 
definitions and rules are necesary to arrive from one thorem to 
another, that is, we can discard those that are not used in the 
particular derivation of a given theorem.

This fact makes mathematics very suitable for organizing the relationships
among theorems in a network structure in which each node is connected
to other nodes only if they have some relation. Relations 
can be of various types, the fastes to think about is 
dependence, that is, when in order to prove something you 
have to make use of other thing.

\paragraph{Optimizing Organization: discarding unsed premisses}
Again, in maths, it is easy to discard unused premisses which makes it
easy to model the interelashionships among concepts as a network.

In other sciences, it's sometimes not as easy to discard premisses
since the intrinsic structure of them is different from maths.
In fact, the very way in which the formal sciences and the natural
ones advances is quite different.

\paragraph{Formal vs. Natural Sciences}
On the side of the formal sciences, you propose certain definitions, 
then postulate certain axioms and finaly use formal rules to derive 
consequences, and a consequence is said to be true if it can be
derived in a recognizable manner from the definitions and axioms.

In the natural sciences however, the truth of a proposition has
to based on the available evidence of the event that the proposition
is describing. This kind of evidence has downsides. The evidence may
not be conclusive due to different interpretantions of the facts, or 
other factors...



\section{Problem: The exponential function}



